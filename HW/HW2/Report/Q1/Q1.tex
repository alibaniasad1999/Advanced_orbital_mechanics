\section{Question 1}

Spacecraft position in ITRF coordinates is given by
$$
\boldsymbol{\mathrm{r}} = 
\begin{bmatrix}
    6789 & 6893 & 7035
\end{bmatrix}^{\mathrm{T}}_{km}
$$

\subsection{part a}
Find Latitude and Longitude. For this purpose used algorithm 12 of Valado's book. This algorithm is implemented in the function 'latlon.py' in the 'code/Q1' folder. The function takes the spacecraft position vector as input and returns the latitude and longitude in degrees. The iteration ended when the difference is smaller than 1e-10. The results are:

\begin{table}[H]
    \caption{Results of part a}
    \begin{center}
        \begin{tabular}{|c|c|}
            \hline
            \Tstrut
            \textbf{Variables} & \textbf{Values} \\
            \hline
            Latitude & \ang{36.12} \\
            Longitude & \ang{45.43} \\
            $h_{ellp}$ & 5591.51$_{km}$ \\
            \hline
            \end{tabular}
    \end{center}
\end{table}

\subsection{part b}
In this part, we used the astropy package to find the position vector in the GCRF coordination system. The Python code for this can be found in the 'code/Q1' folder in the Jupyter Notebook file. Position vector in GCRF:
$$
\boldsymbol{\mathrm{r}} = 
\begin{bmatrix}
    -862.54 & -9634.75 & 7037.25
\end{bmatrix}^{\mathrm{T}}_{km}
$$

\subsection{part c}
In this part, we used the astropy package to find GMST($\theta_{GMST}$) and LST($\theta_{LST}$) The Python code for this can be found in the 'code/Q1' folder in the Jupyter Notebook file. The results are:

\begin{table}[H]
    \caption{Results of part c}
    \begin{center}
        \begin{tabular}{|c|c|}
            \hline
            \Tstrut
            \textbf{GMST($\theta_{GMST}$)} & \textbf{LST($\theta_{LST}$)} \\
            \hline
            \ang{112.78} & \ang{149.78} \\
            \hline
            \end{tabular}
    \end{center}
\end{table}