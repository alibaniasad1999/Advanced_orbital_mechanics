\section{Question 1}
$$
h = 200_{km} \to r = R_e +h = 6378.137 + 200 = 6578.1137,\quad \mu = 3.986  \! \times \! 10^{14}_{m^3/s^2} = 3.986  \! \times \! 10^{5}_{km^3/s^2}
$$
The orbit is circular.

\subsection{Part a}
\begin{equation}
    T = 2\pi \sqrt{\dfrac{r^3}{\mu}} = 2\pi \sqrt{\dfrac{6578.1137^3}{3.986  \! \times \! 10^{5}}} = 5309.62_{\sec}
\end{equation}

\begin{equation}
    T\omega = 2\pi \to \omega = \dfrac{2\pi}{T} = \dfrac{2\pi}{5309.62} = 0.00118_{rad/\sec} 
\end{equation}

\subsection{Part b}

\begin{equation}
    v = \sqrt{\dfrac{\mu}{r}} =  \sqrt{\dfrac{3.986  \! \times \! 10^{5}}{6578.1137}} = 7.78_{km/\sec}
\end{equation}
The new velocity is calculated as:
\begin{equation}
    v_{new} = v + 0.5_{km.\sec} = 8.28_{km.\sec}
\end{equation}
% because:
% \begin{equation}
%     \sqrt{\dfrac{\mu}{r}} = 7.78_{km/\sec} < v_{new} < \sqrt{\dfrac{2\mu}{r}} = 11.01_{km.\sec}
% \end{equation}
% orbit is elliptical.
Assume the new orbit is circular just changed altitude and has a new velocity.
\begin{equation}
    v_{new} = \sqrt{\dfrac{\mu}{r_{new}}} \to r_{new} = \dfrac{\mu}{v_{new}^2} = 5808_{km}
\end{equation}

\begin{equation}
    T = 2\pi \sqrt{\dfrac{r_{new}^3}{\mu}} = 2\pi \sqrt{\dfrac{5808^3}{3.986  \! \times \! 10^{5}}} = 4405.08_{\sec}
\end{equation}

\begin{equation}
    T_{new}\omega_{new} = 2\pi \to \omega_{new} = \dfrac{2\pi}{T_{new}} = \dfrac{2\pi}{4405.08} = 0.00143_{rad/\sec} 
\end{equation}

$r_{new}$ is smaller than the earth's radius.
